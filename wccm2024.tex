\documentclass{wccm2024}

\usepackage{graphicx}
\usepackage{amsmath}
\usepackage{amsfonts}
\usepackage{amssymb}
\usepackage{times}
\title{ON THE ITERATIVE SOLUTION OF SADDLE POINT PROBLEMS USING A SYMMETRIC POSITIVE DEFINITE PRECONDITIONER}

\author{PHILIPPE DEVLOO$^{1}$, GIOVANE AVANCINI$^{2}$ AND MARINA MENEGHEL$^{3}$}

\heading{Philippe Devloo, Giovane Avancini and Marina Meneghel}

\address{$^{1}$ State University of Campinas, Faculty of Civil Engineering\\
Av. Albert Einstein, 901 - Cidade Universitária, Campinas - SP, 13083-852\\
phil@unicamp.br
\and
$^{2}$ State University of Campinas, Faculty of Civil Engineering\\
Av. Albert Einstein, 901 - Cidade Universitária, Campinas - SP, 13083-852\\
giovanea@unicamp.br
\and
$^{3}$ State University of Campinas, Faculty of Civil Engineering\\
Av. Albert Einstein, 901 - Cidade Universitária, Campinas - SP, 13083-852\\
m240534@dac.unicamp.br}

\keywords{Iterative method, saddle-point problem, positive-definite preconditioner, H(div) approximation}

\abstract{Saddle point problems frequently appear in many mathematical and engineering applications. Most systems of partial differential equations with constraints give rise to saddle point linear systems. Typical examples include mixed finite element formulations to solve fluid flows and/or elasticity problems under full incompressibility. The inversion of saddle point problems is challenging due to inherent numerical instability in the direct inversion methods. Many direct and iterative methods have been proposed to overcome this challenges, such as the Schur complement and the Uzawa’s method. In the context of mixed finite element for incompressible flows using stable H(div)-L2 spaces for velocity and pressure, we propose an iterative method that can effectively solve a saddle point problem iteratively by summing a small compressibility to the original matrix. The preconditioning matrix is symmetric positive-definite, which allows the usage of Cholesky decomposition and/or CG-like iterative solvers to compute the incremental solution for the velocities unknowns. A procedure to compute the average pressure of each element of the incompressible problem is developed using the unbalanced fluxes caused by the compressibility perturbation. The average is updated during the iterative process as a function of the velocity increment at each iteration.}

\begin{document}
\thispagestyle{empty}

\section{INTRODUCTION}

All participants whose Abstract has been accepted for presentation are kindly invited to submit the full paper via the web page of the Congress before \textbf{May 31, 2024}.

Conference Proceedings (including full papers only) will be published on Scipedia with DOI number after the congress and will be submitted for indexation to SCOPUS database. \textbf{Submission of the full paper is not mandatory}.

The full paper should be written following the format of macros for submission. The file has to be translated into Portable Document Format (PDF) before submission via the Congress site. The organizers do not commit themselves to include in the Proceedings any full paper received later than the above-mentioned deadline.

The corresponding author should be the speaker, and is expected to register and pay his registration fee during the advance period (before March 31, 2024) for the full paper to be included in the technical program of the Conference.


\section{GENERAL SPECIFICATIONS}

The Full Paper must be written in English within a printing box of 16cm x 21cm, centered in the page. The Full Paper including figures, tables and references must have a minimum length of 6 pages and must not exceed 12 pages. Maximum file size is 4 MB.

\section{TITLE, AUTHORS, AFFILIATION, KEY WORDS}

The first page must contain the Title, Author(s), Affiliation(s),
Key words and the Summary. The Introduction must begin immediately
below, following the format of this template.

\subsection{Title}

The title should be written centered, in 14pt, boldface Roman, all
capital letters. It should be single spaced if the title is more
than one line long.

\subsection{Author}

The author's name should include first name, middle initial and
surname. It should be written centered, in 12pt boldface Roman,
12pt below the title.

\subsection{Affiliation}

Author's affiliation should be written centered, in 11pt Roman,
12pt below the list of authors. A 12pt space should separate two
different affiliations.

\subsection{Key words}

Please, write no more than six key words. They should be written
left aligned, in 12pt Roman, and the line must begin with the
words {\bf Key words}: boldfaced. A 12pt space should separate the
key words from the affiliations.

\subsection {Abstract (optional)}

Use 12pt Italic Roman for the abstract. The word {\bf Abstract} must
be set in boldface, not italicized, at the beginning of the first
line. The text should be justified and separated 12pt from the key
words, as shown in the first page of these instructions.

\section{HEADINGS}

\subsection{Main headings}

The main headings should be written left aligned, in 12pt,
boldface and all capital Roman letters. There should be a 12pt
space before and 6pt after the main headings.

\subsection {Secondary headings}

Secondary headings should be written left aligned, 12 pt, boldface
Roman, with an initial capital for first word only. There should
be a 12pt space before and 6pt after the secondary headings.

\section{EDITORIAL HEADING}

The first page has to include the Editorial Heading, as shown in
the first page of these instructions. Successive pages will
include the name of the authors.

\section{TEXT}

The normal text should be written single-spaced, justified, using
12pt (Times New) Roman in one column. The first line of each
paragraph must be indented 0.5cm. There is not inter-paragraph
spacing.

\section{PAGE NUMBERS}

In order to organize the Full Paper, it is better to number
the pages. Page numbers are not included in the printing box.

\section{FIGURES}

All figures should be numbered consecutively and captioned. The
caption title should be written centered, in 10pt Roman, with
upper and lower case letters.


A 6pt space should separate the figure from the caption, and a
12pt space should separate the upper part of the figure and the
bottom of the caption from the surrounding text.

Figures may be included in the text or added at the bottom of the Full Paper.

\section{EQUATIONS}

A displayed equation is numbered, using Arabic numbers in
parentheses. It should be centered, leaving a 6pt space above and
below to separate it from the surrounding text.

The following example is a single line equation:
\vskip-.6cm
\begin{eqnarray}
Ax = b
\end{eqnarray}

The next example is a multi-line equation:
\vskip-.6cm
\begin{eqnarray}
Ax = b \\
Ax = b \nonumber
\end{eqnarray}

\section{TABLES}

All tables should be numbered consecutively and include caption (10pt Roman, upper and lower case letters).

\begin{table}[h!]
\caption{Example of the construction of one table}
\begin{center}
\begin{tabular}{*{3}{c}}
\hline
C11 & C12 & C13 \\
\hline
C21 & C22 & C23 \\
\hline
C31 & C32 & C33 \\
\hline
C41 & C42 & C43 \\
\hline
C51 & C52 & C53 \\
\hline
\end{tabular}
\end{center}
\end{table}

A 6pt space should separate the table from the caption, and a 12pt
space should separate the table from the surrounding text.

\section{FORMAT OF REFERENCES}

References should be quoted in the text by superscript
numbers \cite{Zienkiewicz,Idelsohn} and grouped together at the end of the Extended
Abstract in numerical order as shown in these instructions.

\section{CONCLUSIONS}

\begin{itemize}
\item[-] Full Papers in format for publication should be submitted electronically via the web page of the Congress, before \textbf{May 31, 2024}. The file must be converted to Portable Document Format (PDF) before submission. The maximum size of the file is 4 Mb.

\item[-] Recall that the speaker is expected to pay his registration fee by \textbf{March 31, 2024} for the presentation to be included in the final programme of the Conference.
\end{itemize}

\begin{thebibliography}{99}
\bibitem{Zienkiewicz}  Zienkiewicz, O.C. and  Taylor, R.L. \textit{The finite element method}. McGraw Hill,
Vol. I., (1989), Vol. II., (1991).
\bibitem{Idelsohn} Idelsohn, S.R. and O\~{n}ate, E. Finite element and finite volumes. Two good friends.
\textit{Int. J. Num. Meth. Engng.} (1994) \textbf{37}:3323--3341.
\end{thebibliography}

\end{document}


