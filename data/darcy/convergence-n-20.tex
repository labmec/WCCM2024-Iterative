\documentclass[margin=0cm]{standalone}
\usepackage{amsmath}
\usepackage{tikz}
\usepackage{pgfplots}

\pgfplotsset{compat=newest}

\usetikzlibrary{calc}
\usetikzlibrary{plotmarks}

%%% START MACRO FOR ANNOTATION OF TRIANGLE WITH SLOPE %%%.
\newcommand{\logLogSlopeTriangle}[5]
{
    % #1. Relative offset in x direction.
    % #2. Width in x direction, so xA-xB.
    % #3. Relative offset in y direction.
    % #4. Slope d(y)/d(log10(x)).
    % #5. Plot options.

    \pgfplotsextra
    {
        \pgfkeysgetvalue{/pgfplots/xmin}{\xmin}
        \pgfkeysgetvalue{/pgfplots/xmax}{\xmax}
        \pgfkeysgetvalue{/pgfplots/ymin}{\ymin}
        \pgfkeysgetvalue{/pgfplots/ymax}{\ymax}

        % Calculate auxilliary quantities, in relative sense.
        \pgfmathsetmacro{\xArel}{#1}
        \pgfmathsetmacro{\yArel}{#3}
        \pgfmathsetmacro{\xBrel}{#1-#2}
        \pgfmathsetmacro{\yBrel}{\yArel}
        \pgfmathsetmacro{\xCrel}{\xArel}
        %\pgfmathsetmacro{\yCrel}{ln(\yC/exp(\ymin))/ln(exp(\ymax)/exp(\ymin))} % REPLACE THIS EXPRESSION WITH AN EXPRESSION INDEPENDENT OF \yC TO PREVENT THE 'DIMENSION TOO LARGE' ERROR.

        \pgfmathsetmacro{\lnxB}{\xmin*(1-(#1-#2))+\xmax*(#1-#2)} % in [xmin,xmax].
        \pgfmathsetmacro{\lnxA}{\xmin*(1-#1)+\xmax*#1} % in [xmin,xmax].
        \pgfmathsetmacro{\lnyA}{\ymin*(1-#3)+\ymax*#3} % in [ymin,ymax].
        \pgfmathsetmacro{\lnyC}{\lnyA+#4*(\lnxA-\lnxB)}
        \pgfmathsetmacro{\yCrel}{\lnyC-\ymin)/(\ymax-\ymin)} % THE IMPROVED EXPRESSION WITHOUT 'DIMENSION TOO LARGE' ERROR.

        % Define coordinates for \draw. MIND THE 'rel axis cs' as opposed to the 'axis cs'.
        \coordinate (A) at (rel axis cs:\xArel,\yArel);
        \coordinate (B) at (rel axis cs:\xBrel,\yBrel);
        \coordinate (C) at (rel axis cs:\xCrel,\yCrel);

        % Draw slope triangle.
        \draw[#5]   (A)-- node[pos=0.5,anchor=north] {1}
                    (B)-- 
                    (C)-- node[pos=0.5,anchor=west] {#4}
                    cycle;
    }
}
%%% END MACRO FOR ANNOTATION OF TRIANGLE WITH SLOPE %%%.

\begin{document}
    \begin{tikzpicture}
        \begin{semilogyaxis}[
	        xlabel=Iteration $k$,
	        ylabel=${\lVert \mathbf{r}^k \rVert}$,
            xmin=0, xmax=25,
            grid style={dashed},
            grid=major,
            % legend style=
            % {
            %     at={(0.95,0.95)},
            %     anchor=north east
            % }
            legend pos = outer north east
            ]
        \addplot[color=red,mark=*, restrict x to domain=1:7] table[x index=0, y index=1] {convergence-n-20.dat};
        \addplot[color=blue,mark=*, restrict x to domain=1:4] table[x index=0, y index=2] {convergence-n-20.dat};
        \addplot[color=green,mark=*, restrict x to domain=1:4] table[x index=0, y index=3] {convergence-n-20.dat};
        \addplot[color=orange,mark=*, restrict x to domain=1:3] table[x index=0, y index=4] {convergence-n-20.dat};
        \addplot[color=purple,mark=*, restrict x to domain=1:25] table[x index=0, y index=5] {convergence-n-20.dat};
        \addplot[color=brown,mark=*, restrict x to domain=1:25] table[x index=0, y index=6] {convergence-n-20.dat};
        \addplot[color=black,mark=*, restrict x to domain=1:25] table[x index=0, y index=7] {convergence-n-20.dat};

        \legend{$\alpha=10^{0}$, $\alpha=10^{-1}$, $\alpha=10^{-2}$, $\alpha=10^{-3}$, $\alpha=10^{-4}$, $\alpha=10^{-5}$, $\alpha=10^{-6}$}
        \end{semilogyaxis}
    \end{tikzpicture}
\end{document}
